\documentclass[letterpaper]{article}
\usepackage{xeCJK}
\usepackage[breaklinks,urlbordercolor={1 1 1}]{hyperref}
\usepackage{geometry}
\usepackage{tabularx}
\usepackage[resetlabels]{multibib}
\usepackage[normalem]{ulem}
\usepackage{xcolor}
\newcites{A}{\textsc{论文}}

% Comment the following lines to use the default Computer Modern font
% instead of the Palatino font provided by the mathpazo package.
% Remove the 'osf' bit if you don't like the old style figures.
\usepackage[T1]{fontenc}
\usepackage[sc,osf]{mathpazo}

% Set your name here
\def\name{\textbf{U08M11002 信号与系统}}

% Replace this with a link to your CV if you like, or set it empty
% (as in \def\footerlink{}) to remove the link in the footer:
\def\footerlink{}

% The following metadata will show up in the PDF properties
\hypersetup{
  colorlinks = true,
  urlcolor = black,
  pdfauthor = {\name},
  pdftitle = {\name: Curriculum Vitae},
  pdfsubject = {Curriculum Vitae},
  pdfpagemode = UseNone
}

\geometry{
  body={6.5in, 9in},
  left=1.0in,
  top=1.25in
}

% Customize page headers
\pagestyle{myheadings}
\markright{\name}
\thispagestyle{empty}

% Custom section fonts
\usepackage{sectsty}
\sectionfont{\rmfamily\mdseries\Large}
\subsectionfont{\rmfamily\mdseries\itshape\large}

% Other possible font commands include:
% \ttfamily for teletype,
% \sffamily for sans serif,
% \bfseries for bold,
% \scshape for small caps,
% \normalsize, \large, \Large, \LARGE sizes.

% Don't indent paragraphs.
\setlength\parindent{0em}

% Make lists without bullets
\renewenvironment{itemize}{
  \begin{list}{}{
    \setlength{\leftmargin}{1.5em}
  }
}{
  \end{list}
}

\begin{document}

% Place name at left
{\huge \name}

% Alternatively, print name centered and bold:
%\centerline{\huge \name}

\vspace{0.1in}

\begin{minipage}{0.8\linewidth}
	2023 年秋季学期
\end{minipage}

%---------------------------------------------------------------
%	课程信息
%---------------------------------------------------------------

\section*{\textsc{课程信息}}
\begin{tabular}{ll}
学分/学时 & 4学分/64学时 \vspace{.5em}\\
上课时间 & 	每周三、周五 11-12 节 19:00-20:40 \vspace{.5em}\\
上课地点 &	教西 C307 \vspace{.5em}\\
课程 QQ 群 & 910379259  \vspace{.5em}\\
先修课程 & 高等数学 \vspace{.5em}\\
课程材料网站 & https://github.com/aprilabdotdev/U08M11002 \vspace{.5em}\\
\end{tabular}

%---------------------------------------------------------------
%	教师信息
%---------------------------------------------------------------

\section*{\textsc{教师信息}}
\begin{tabular}{ll}
授课教师 & 	秦雨潇 \vspace{.5em}\\
办公室 &	电子信息学院大楼 133 / 259 \vspace{.5em}\\
邮箱 & yuxiao.qin@nwpu.edu.cn  \vspace{.5em}\\
QQ & 见 QQ 群  $\leftarrow$ 可能是你们最喜欢的方式 \vspace{.5em}\\
\end{tabular}


%---------------------------------------------------------------
%	分数
%---------------------------------------------------------------

\section*{\textsc{\textcolor{red}{课程分数组成}}}
\textbf{[15\%] MOOC + [15\%]平时 +  [70\%] 期末考试} \vspace{.5em}\\
其中,MOOC 的部分请在 https://www.icourse163.org/course/NWPU-1002989009 完成。注意,请参加“\textbf{第十一次开课}”。

%---------------------------------------------------------------
%	教材及参考书目
%---------------------------------------------------------------

\section*{\textsc{教材及参考书目}}
\begin{tabular}{ll}
教材    & 段哲民等,《信号与系统》(第四版),电子工业出版社,2020 \vspace{.5em}\\
参考书目 & 吴大正等,《信号与线性系统分析》(第五版),高等教育出版社, 2019 \vspace{.5em} \\
& Oppenheim等著,刘树棠译,《Signals and Systems》(第二版), 电子工业出版社,2012\vspace{.5em}\\
参考慕课 & 西安电子科技大学(郭宝龙)、Oppenheim(MIT)信号与系统课程
\end{tabular}

%---------------------------------------------------------------
%	教学目标
%---------------------------------------------------------------
\section*{\textsc{教学目标}}
\quad\quad 信号与系统课程是电子、电气、通信、计算机、信息处理等电类专业本科生的一门重要的技术基础课程。它是以大学物理、高等数学、工程数学、电路分析为基础,同时又是后续技术基础课程和专业课的基础。在教学中具有承前启后、继往开来的作用。是学生合理知识结构的重要组成部分,在发展智力、培养能力和良好的非智力素质方面,起着极为重要的作用。本课程的主要内容有:信号与系统的基本概念、连续系统时域分析、连续信号频域分析、连续系统频域分析、连续系统的复频域分析、复频域系统函数与系统模拟、离散信号与系统时域分析、离散信号与系统Z域分析、状态变量法等九章内容。\par
\quad\quad 通过该课程的学习,掌握信号与系统的基本概念,连续信号与系统时域、频域、复频域的分析方法;离散信号与系统时域、复频域的分析方法;连续系统与离散系统的状态变量分析法。提高学生理论联系实际、分析解决复杂工程问题的能力,为进一步学习信号处理、通信理论、自动控制、人工智能等课程打下良好的基础。

%---------------------------------------------------------------
%	教学大纲
%---------------------------------------------------------------
\section*{\textsc{教学大纲}}
第一章 \quad 信号与系统的基本概念(6学时)\par
\quad\quad 1.1信号的描述与分类\par
\quad\quad 1.2常用的连续时间信号及其时域特性\par
\quad\quad 1.3连续时间信号时域变换与运算\par
\quad\quad 1.4系统的定义与分类\par
\quad\quad 1.5线性时不变系统的性质\par
\quad\quad 要求:(1)了解信号与系统的基本概念与定义,会画出信号的波形;(2)了解常用基本信号的时域描述方法及其特点与性质,并会应用这些性质;(3)深刻理解信号的时域分解及其变换与运算的方法,并会求解;(4)深刻理解线性时不变系统的定义与性质,并会应用这些性质。\par
\vspace{1em}
第二章 \quad 连续系统时域分析(6学时)\par
\quad\quad 2.1经典时域分析方法\par
\quad\quad 2.2微分方程的微分算子表示\par
\quad\quad 2.3零输入响应与零状态响应\par
\quad\quad 2.4系统的冲激响应与阶跃响应\par
\quad\quad 2.5卷积积分\par
\quad\quad 2.6求系统零状态响应的卷积积分法\par
\quad\quad 要求:(1)了解从物理模型建立连续时间系统数学模型的方法;(2)掌握常系数线性微分方程的经典解法;(3)深刻理解全响应的三种分解方法;(4)深刻理解系统单位冲激响应与阶跃响应的概念,并会求解;(5)掌握零输入响应的概念,会根据微分方程的特征根与已知的系统的初始条件求系统的零输入响应;(6)掌握卷积积分的概念及其性质;(7)掌握零状态响应的概念,会应用卷积积分求线性时不变系统的零状态响应;(8)会求系统的全响应。\par
\vspace{1em}
第三章 \quad 连续信号频域分析(8学时)\par
\quad\quad 3.1信号的完备正交函数集表示\par
\quad\quad 3.2连续时间周期信号的傅里叶级数表示\par
\quad\quad 3.3周期信号的频谱\par
\quad\quad 3.4非周期信号的频谱\par
\quad\quad 3.5傅里叶变换的基本性质\par
\quad\quad 3.6周期信号的傅里叶变换\par
\quad\quad 3.7功率信号、能量信号及其功率谱与能量谱\par
\quad\quad 要求:(1)了解函数正交的条件及完备正交函数集的概念。掌握信号表示为正交函数的基本思想和常见的正交函数集;(2)能用傅里叶级数的定义式、性质及周期信号的傅里叶变换,求解周期信号的频谱、频谱宽度,画频谱图;深刻理解周期信号频谱的特点;(3)能利用傅里叶变换的定义式、性质,求解非周期信号的频谱,画频谱图,求信号的频谱宽度;会对信号进行正、反傅里叶变换;(4)深刻理解和掌握周期信号的傅里叶变换及周期信号与非周期信号傅里叶变换之间的关系;(5)深刻理解功率信号与功率谱、能量信号与能量谱的概念,会在时域与频域两个域中求解功率信号的功率与能量信号的能量。\par
\vspace{1em}
第四章 \quad\quad 连续系统频域分析(6学时)\par
\quad\quad 4.1系统对非正弦周期信号的响应\par
\quad\quad 4.2系统对非周期信号的响应\par
\quad\quad 4.3频域系统函数\par
\quad\quad 4.4信号传输失真及无失真传输条件\par
\quad\quad 4.5理想低通滤波器及其响应\par
\quad\quad 4.6抽样信号与抽样定理\par
\quad\quad 4.7调制与解调\par
\quad\quad 要求:(1)深刻理解频域系统函数的定义,物理意义、求法与应用,并会求解;(2)会求解非正弦周期信号激励下系统的稳态响应;(3)会求解非周期信号激励下系统的零状态响应与全响应;(4)深刻理解理想低通滤波器的定义、传输特性(冲激响应与阶跃响应)及其上升时间的意义;(5)了解信号无失真传输的条件;(6)深刻理解和掌握抽样信号的频谱及其求解;深刻理解和掌握抽样定理;(7)了解调制与解调的基本原理与应用。\par
\vspace{1em}
第五章 \quad 连续系统的复频域分析(6学时)\par
\quad\quad 5.1拉普拉斯变换 \par
\quad\quad 5.2基尔霍夫定律与电路元件的复频域形式\par
\quad\quad 5.3线性系统复频域分析法\par
\quad\quad 5.4拉普拉斯变换与傅里叶变换的关系\par
\quad\quad 要求:(1)深刻理解拉普拉斯变换的定义式,收敛域及基本性质;会根据拉普拉斯变换的定义式及基本性质,求一些常用信号的拉普拉斯变换;(2)正确理解拉普拉斯变换的性质(特别是时移性,频移性,时域微分,频域微分,初值定理,终值定理等性质)及其应用条件。(3)能应用部分分式法和留数法,求一些像函数的拉普拉斯反变换;(4)掌握S域中电路KCL,KVL的表示形式及电路元件的伏安关系;能根据时域电路模型正确地画出S域电路模型;(5)能利用单边拉普拉斯变换与S域电路模型,求冲击响应与阶跃响应;(6)掌握拉普拉斯变换与傅立叶变换的关系。\par
\vspace{1em}
第六章 \quad 复频域系统函数与系统模拟(8学时)\par
\quad\quad 6.1复频域系统函数及其零、极点图\par
\quad\quad 6.2系统函数的应用\par
\quad\quad 6.3连续系统的模拟图与框图\par
\quad\quad 6.4连续系统的信号流图与梅森公式\par
\quad\quad 6.5连续系统的稳定性及其判定\par
\quad\quad 要求:(1)深刻理解复频域系统函数的定义及物理意义,会用多种方法进行求解;(2)了解复频域系统函数的一般表达形式及的零点与极点概念,会画零极点图,并会根据零极点图求解;(3)深刻理解系统模拟与信号流图的意义;能根据系统的微分方程或画出系统直接形式、并联形式、级联形式的模拟图与信号流图;能根据模拟图或信号流图按梅森公式求出系统函数;能根据电路图或模拟图或框图,画出相应的信号流图;(4)深刻理解系统稳定性的意义,会根据系统函数的极点分布与罗斯判据,判定系统是否为稳定系统。\par
\vspace{1em}
第七章 \quad 离散信号与系统时域分析(8学时)\par
\quad\quad 7.1离散信号\par
\quad\quad 7.2离散时间信号的时域运算\par
\quad\quad 7.3常用的离散时间信号\par
\quad\quad 7.4离散系统及其数学描述\par
\quad\quad 7.5离散时间系统的时域经典分析\par
\quad\quad 7.6离散系统的单位序列响应\par
\quad\quad 7.7离散系统的卷积和分析\par
\quad\quad 要求:(1)掌握离散信号的定义及其描述;离散信号的能量和功率;(2)掌握离散信号的时域运算和变换;掌握离散信号分解为单位序列信号的方法;(3)掌握常用离散信号的定义和时域特性,理解常用离散信号与连续信号的区别与联系。(4)掌握离散时间系统的描述及线性时不变离散系统的性质;了解离散系统数学模型的建立;了解时间域用差分方程和传输算子模拟系统的方法;会求系统的自然频率。(5)掌握离散时间系统的时域经典分析方法,会求解差分方程(齐次和非齐次);深刻理解系统的初始状态、差分方程的初始条件和响应的初始值的含义,并能够确定求解差分方程的初始条件;掌握零输入响应和零状态响应的求解方法;深刻理解全响应的三种分解形式;(6)掌握离散时间系统的单位序列响应的求解方法:迭代法、等效初值法和传输算子法。(7)掌握卷积和的定义及性质;掌握离散信号的卷积和运算方法;会用卷积和求解系统的零状态响应。\par
\vspace{1em}
第八章 \quad 离散信号与系统Z域分析(8学时)\par
\quad\quad 8.1离散信号的Z变换\par
\quad\quad 8.2 Z变换的基本性质\par
\quad\quad 8.3 Z反变换\par
\quad\quad 8.4利用Z变换求解离散系统的响应\par
\quad\quad 8.5 Z域系统函数\par
\quad\quad 8.6系统函数的零、极点分布对系统特性的影响\par
\quad\quad 8.7用朱利准则判断离散系统的稳定性\par
\quad\quad 要求:(1)深刻理解并掌握Z变换的定义及其收敛域;理解不同形式序列的Z变换对应的收敛域;掌握常用离散信号的Z变换及其收敛域;熟练掌握Z变换的性质,并注意单边Z变换与双边Z变换在某些性质的区别以及某些性质的应用条件;熟练掌握利用性质求序列的Z变换;(2)了解Z变换与离散序列的傅立叶变换之间的关系,由Z变换会求离散序列的傅立叶变换;了解Z变换与连续信号的拉氏变换之间的关系,了解s平面与z平面的映射关系;(3)掌握Z逆变换的求解方法,特别是部分分式法;(4)掌握利用Z变换求解系统的差分方程,即求解系统的零输入响应、零状态响应和全响应;(5)掌握系统函数的定义及其物理意义;理解系统函数的零极点的含义,会画零极点图;理解并掌握系统函数的零极点分布与系统单位序列响应的关系;掌握各种系统函数求解方法;(6)掌握系统函数的应用,特别是用系统函数分析系统的因果性、稳定性,求解差分方程,求解系统的频率特性和正弦稳态响应,用系统函数模拟系统;掌握朱利准则。\par
\vspace{1em}
第九章 \quad 状态变量法(8学时)\par
\quad\quad 9.1基本概念与定义\par
\quad\quad 9.2连续系统状态方程与输出方程的建立\par
\quad\quad 9.3连续系统状态方程与输出方程的s域解法\par
\quad\quad 9.4连续系统状态方程与输出方程的时域解法\par
\quad\quad 9.5离散系统状态变量分析\par
\quad\quad 9.6由状态方程判断系统的稳定性\par
\quad\quad 要求:(1)掌握系统在状态空间中的描述:状态,状态变量,状态空间,状态方程与输出方程;(2)掌握系统状态方程的建立,包括连续系统和离散系统状态变量的选择、状态方程与输出方程的建立及其向量表示法;(3)掌握连续时间系统状态方程的时域解法:一阶向量状态差分方程的解状态过渡矩阵指数函数的计算,输出方程的解,单位冲激响应矩阵及其转移函数矩阵间的关系;(4)掌握连续时间系统状态方程的复频域解法:应用拉普拉斯变换法求解一阶向量状态微分方程与输出方程;多输入多输出系统的转移函数矩阵,A矩阵的特征根和系统的自然频率;(5)掌握离散时间系统状态方程的时域解法:一阶向量状态差分方程的解,离散时间系统的状态过渡矩阵,输出方程的解,单位函数响应矩阵及其与转移函数矩阵间的关系;(6)掌握离散时间系统状态方程的Z变换解法:应用Z变换法解一阶向量状态差分方程与输出方程;离散时间系统的转移函数矩阵。\par
\vspace{1em}


%---------------------------------------------------------------
%	学术道德
%---------------------------------------------------------------
\section*{\textsc{学术道德}}
\quad\quad 请严格遵守《\href{https://renshi.nwpu.edu.cn/info/1441/4824.htm}{西北工业大学学术道德规范及管理办法}》。\par


% Footer
\begin{center}
  \begin{footnotesize}
    Last updated: \today \\
%    \href{\footerlink}{\texttt{\footerlink}}
  \end{footnotesize}
\end{center}

\end{document}
