\documentclass[12pt, a4paper, oneside]{ctexart}
\usepackage{amsmath, amsthm, amssymb, bm, color, framed, graphicx, hyperref, mathrsfs, float, caption,subfigure}
\usepackage[justification=centering]{caption}

% multi-column
\usepackage{tasks}
% itemize
\NewTasksEnvironment[label=(\arabic*), label-width=3ex]{exercise}

\everymath{\displaystyle}

\title{\textbf{第八次作业}}
\author{U08M11002 Spring 2022}
\date{提交截止日期:北京时间2022年6月1日}
\linespread{1}
\definecolor{shadecolor}{RGB}{241, 241, 255}

\newcounter{problemname}
\newenvironment{problem}{\stepcounter{problemname}\par\noindent\textbf{题目\arabic{problemname}. }}{\\\par}
\newenvironment{warning}{\begin{shaded}\par\noindent\textbf{提交作业方式:}}{\end{shaded}\par}

\begin{document}
	
	\maketitle
	
	\begin{warning}
		具体提交方式请以 QQ 群里助教的通知为准。
		\begin{enumerate}
			\item 为了你自己复习需要,\textbf{建议上交前自行扫描备份}。
		\end{enumerate}
	\end{warning}
	
	\hspace{1em}
	
	
	\begin{problem}
		已知系统的激励为$f(t) = (e^{-t} + e^{-3t})U(t)$,系统的零状态响应为$y(t) = (2e^{-t} - 2e^{-4t})U(t)$。
		\begin{exercise}
			\task 求系统的单位冲激响应$h(t)$;
			\task 求系统的微分方程;
		\end{exercise}
		
		\quad
	\end{problem}
	
	
	\begin{problem}
		已知系统函数$H(s) = \frac{s+3}{s^{2} + 3s + 2}$,激励$f(t) = e^{-3t}U(t)$,初始状态$y(0^{-}) = 1$,$y^{'}(0^{-}) = 2$。求系统的全响应$y(t)$,零输入响应$y_{x}(t)$,零状态响应$y_{f}(t)$,并确定其自由响应与强迫响应分量。
		\quad
	\end{problem}
	
	\begin{problem}
		已知$H(s) = \frac{3s}{s^{3} + 4s^{2} + 6s + 4}$,试画出直接形式、并联形式、级联形式的信号流图。
		\quad
	\end{problem}
	
	
	\begin{problem}
		求下列序列的卷积和$y(k) = f_{1}(k) * f_{2}(k)$。
		\begin{exercise}
			\task $f_{1}(k) = (\frac{1}{2})^{\textbar k \textbar}$,$f_{2}(k) = 1$;
			\task $f_{1}(k) = \{\mathop{2}\limits_{\uparrow},2,1,-1 \}$,$f_{2}(k) = \{\mathop{0}\limits_{\uparrow},1,4,-2\}$;
			\task $f_{1}(k) = (2)^{k+1}U(k+1)$,$f_2(k) = \delta(2 - k) + U(k)$;
			\task $f_1(k) = (0.5)^{k}U(k)$,$f_2(k) = U(-k)$;
		\end{exercise}
		\quad
	\end{problem}
	

	\begin{problem}
		求下列卷积和:
		\begin{exercise}
			\task $U(k) * U(k)$;
			\task $(0.25)^{k}U(k) * U(k)$;
			\task $5^{k}U(k) * 3^{k}U(k)$;
			\task $kU(k) * \delta(k - 2)$;
		\end{exercise}
		
		\quad
	\end{problem}
	
	\begin{problem}
		求下列像函数的逆变换。
		\begin{exercise}
			\task $F(z) = \frac{z - 1}{z - 2}$,$\textbar z \textbar > 2$;
			\task$F(z) = \frac{1}{z^{2} + 1}$,$\textbar z \textbar > 1$;
			\task $F(z) = z^{-1} + 5z^{-2} -3z^{-5}$, $\textbar z \textbar > 0$;
			\task$F(z) = \frac{z^{3}}{z - 2}$,$2 < \textbar z \textbar < \infty$;
		\end{exercise}
		\quad
	\end{problem}
	
	\begin{problem}
		用部分分式展开法求反变换。
		\newline
			\centerline{$F(z) = \frac{2z^{3} - 5z^{2} + z + 3}{z^{2} - 3z + 2}$,$2 < \textbar z \textbar < \infty$}
		\quad
	\end{problem}
	
	\begin{problem}
		已知某$\mathrm{LTI}$因果系统的差分方程为:
		\newline
		\centerline{$y(k) - y(k-1) - 2y(k-2) = f(k) + 2f(k - 2)$。}
		利用$z$变换求当$y(-1) = 2$,$y(-2) = -0.5$,激励为$f(k) = U(k)$时,系统的零输入响应和零状态响应。

		\quad
	\end{problem}
	
\end{document}